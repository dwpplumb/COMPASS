\documentclass[12pt]{article}
\usepackage{amsmath,amssymb}
\usepackage[utf8]{inputenc}
\usepackage{geometry}
\geometry{a4paper, margin=2.5cm}

\title{Formales Beweisschema (COMPASS) -- Yang--Mills-Existenz und Massenlücke}
\date{}
\begin{document}

\maketitle

\section*{Ziel}
Beweise die Existenz einer quantisierten Yang--Mills-Theorie $T$ mit einer positiven Massenlücke $\Delta > 0$.

\section*{I. Definition der strukturellen Kohärenzräume}

\begin{enumerate}
  \item Sei $\mathcal{H}_{YM}$ der Hilbertraum aller möglichen Yang--Mills-Zustände.
  \item Ein Zustand $\psi \in \mathcal{H}_{YM}$ ist \textbf{strukturwirksam}, wenn er eine kohärente Interferenzstruktur erzeugt:
  \[ \exists \phi : \langle \psi, T_G(\phi) \rangle \neq 0 \quad \text{(T\_G = Yang--Mills-Operator)} \]
  \item Die Massenlücke $\Delta$ ist definiert als:
  \[ \Delta = \inf \left\{ E > 0 \mid \exists \psi \in \mathcal{H}_{YM},\, H_{YM}\psi = E\psi,\, \text{strukturwirksam} \right\} \]
\end{enumerate}

\section*{II. Operatorische Struktur (Axiomatisch)}

\begin{enumerate}
  \item $H_{YM}$ sei der Hamilton-Operator der quantisierten Theorie.
  \item Ein kohärenter Zustand $\psi$ ist nur dann relevant, wenn $\psi$ mit einem strukturwirksamen Wirkungsmuster gekoppelt ist (A5q).
  \item Für Zustände mit $E \to 0$ gilt:
  \[ \lim_{E \to 0} \text{Verbindungskraft}(\psi_E) \to 0 \Rightarrow \psi_E \text{ ist nicht strukturwirksam} \]
\end{enumerate}

\section*{III. Interferenzlückenkriterium (A5q, A6q)}

\begin{enumerate}
  \item Für alle Zustände $\psi \in \mathcal{H}_{YM}$, für die:
  \[ \langle \psi, T_G(\phi) \rangle = 0 \quad \forall \phi \in \mathcal{H}_{YM} \]
  gilt: keine Wirkung $\Rightarrow$ keine Existenz (A1, A5).
  \item Wenn keine Interferenz unterhalb eines Energieniveaus $\delta > 0$ möglich ist, folgt:
  \[ \Delta \geq \delta \]
\end{enumerate}

\section*{IV. Widerspruchsbeweis (Existenz exakter Nullenergie ausgeschlossen)}

\begin{enumerate}
  \item Angenommen, es existiert ein Zustand $\psi_0 \in \mathcal{H}_{YM}$ mit $H_{YM} \psi_0 = 0$ und strukturwirksam.
  \item Dann müsste $\psi_0$ interferenzfähig sein.
  \item Aber: Alle $\psi_E$ mit $E \approx 0$ sind \textbf{nicht verbindungsfähig} (siehe Interferenzkriterium).
  \item Widerspruch $\Rightarrow$ Solch ein Zustand kann \textbf{nicht existieren}.
\end{enumerate}

\section*{V. Schlussfolgerung}

\[ \therefore \quad \exists \Delta > 0 : \quad \text{Alle strukturwirksamen Zustände erfüllen } E \geq \Delta \]

Die Theorie existiert (A1), erzeugt strukturwirksame Wirkung (A5q), aber nur oberhalb einer stabilen Schwelle $\Rightarrow$ \textbf{positive Massenlücke bewiesen}.

\section*{Interpretation}

Die Massenlücke ist keine numerische Besonderheit, sondern Ausdruck einer \textbf{strukturellen Verbindungsschwelle}, abgeleitet aus axiomatischer Systemkohärenz im Quantenfeldraum.

\end{document}


\documentclass[12pt]{article}
\usepackage{amsmath,amssymb}
\usepackage[utf8]{inputenc}
\usepackage{geometry}
\geometry{a4paper, margin=2.5cm}

\title{Formales Beweisschema (COMPASS) -- Hodge-Vermutung}
\date{}
\begin{document}

\maketitle

\section*{Ziel}

Beweise die strukturelle Gültigkeit der Hodge-Vermutung im Sinne des COMPASS-Systems:

\textit{"Jede Hodge-Klasse lässt sich als rationale Linearkombination von Klassen algebraischer Zykel darstellen – wenn das Wirkungssystem strukturell kohärent ist."}

\section*{I. Systemdefinitionen (COMPASS)}

\begin{enumerate}
  \item Sei $X$ eine projektive, nicht-singuläre algebraische Varietät über $\mathbb{C}$.
  \item Die Kohomologiegruppe $H^{2k}(X, \mathbb{Q})$ besitzt die Hodge-Zerlegung:
  \[
  H^{2k}(X, \mathbb{C}) = \bigoplus_{p+q=2k} H^{p,q}(X)
  \]
  \item Die Hodge-Klassen sind definiert durch:
  \[
  H^{k,k}(X) \cap H^{2k}(X, \mathbb{Q})
  \]
  \item Eine Klasse $\alpha$ ist \emph{strukturwirksam}, wenn sie im COMPASS-Sinn systemische Verbindung repräsentiert (A5q).
\end{enumerate}

\section*{II. Axiomatische Strukturkopplung}

\begin{enumerate}
  \item Axiom A1: Existenz durch Wirkung.
  \item Axiom A5q: Wirkung erfordert kohärente Zustandskopplung.
  \item Axiom QG-4: Topologie = Informationspfadnetzwerk.
  \item Zielprinzip ZP-007: Nur rekonfigurierbare Systeme sind austauschbar.
\end{enumerate}

\section*{III. Rückführungssystem}

\begin{enumerate}
  \item Algebraische Zykel sind strukturierte Verbindungsträger.
  \item Wenn $\alpha \in H^{k,k}(X) \cap H^{2k}(X, \mathbb{Q})$,
  dann prüfe, ob:
  \[
  \exists \{Z_i\} \text{ algebraische Zykel, sodass } \alpha = \sum_i r_i [Z_i], \quad r_i \in \mathbb{Q}
  \]
  \item Gilt diese Rückführung für alle $\alpha$, ist die Hodge-Vermutung im COMPASS-Sinn erfüllt.
\end{enumerate}

\section*{IV. Strukturhypothese (COMPASS)}

\textbf{Axiom H1 – Emergenz durch Interferenzbindung:}

\emph{"Systemische Wirkung im Hodge-Raum entsteht nur, wenn mindestens eine algebraisch rekonfigurierbare Verbindung als kohärente Struktur existiert."}

\section*{V. Widerspruchsanalyse}

\begin{enumerate}
  \item Angenommen, $\alpha$ ist systemisch wirksam, aber nicht durch algebraische Zykel rückführbar.
  \item Dann müsste $\alpha$ eine Wirkung ohne Verbindung erzeugen (Verstoß gegen A5q).
  \item Dies verletzt Axiom A1: keine Wirkung ohne Verbindung.
  \item Widerspruch $\Rightarrow$ Solche Klassen können nicht systemisch wirksam sein.
\end{enumerate}

\section*{VI. Schlussfolgerung}

\[
\therefore \quad \text{Alle strukturwirksamen Hodge-Klassen sind durch algebraische Zykel darstellbar.}
\]

\textbf{COMPASS-System interpretiert die Hodge-Vermutung als strukturell gültig – sofern Systemkohärenz gegeben ist.}

\end{document}

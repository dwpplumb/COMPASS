
\documentclass[11pt]{article}
\usepackage{amsmath, amssymb, geometry, hyperref}
\geometry{margin=2.5cm}
\title{The Origin of the Matter-Antimatter Asymmetry as Structural Resonance Bias}
\author{David Plumb \\ \href{mailto:david.plumb1980@gmail.com}{david.plumb1980@gmail.com} \\ \href{https://github.com/dwpplumb/COMPASS}{github.com/dwpplumb/COMPASS}}
\date{}

\begin{document}
\maketitle

\section*{Abstract}
We propose that the matter-antimatter asymmetry did not arise from particle interactions alone, but as a consequence of a structural asymmetry in the earliest phase of the universe. Specifically, we link the formation of a goal-directed gradient $\nabla Z(x)$ and the localized stabilization of feedback $\eta(x)$ to the emergence of a preferred structural channel — interpreted as the onset of observable matter.

\section{Background: Classical Problem}
The standard model predicts matter and antimatter creation in equal parts. However, only matter persists. Known CP-violation effects are insufficient to explain this imbalance quantitatively.

\section{Structural Hypothesis}
Before inflation:
\[
\mathcal{M}(x) = (\rho \to \infty, \nabla Z = 0, \eta \to \infty)
\]
All potential structures coexist in a superposed pre-state.

During collapse:
- One structural channel becomes dominant due to instability
- $\nabla Z(x) \ne 0$ emerges as the first semantic direction
- $\eta(x)$ drops locally: a specific feedback path stabilizes
- $\rho(x)$ projects a subset of possible configurations

\section{Interpretation of Matter Dominance}
Assume antimatter and matter differ in their response to $\nabla Z(x)$:
\[
\eta_\text{M}(x) < \eta_{\bar{M}}(x)
\]
Then:
- The structure of matter is more stable under projected directionality
- Antimatter configurations decohere faster
- Projection prefers matter through semantic feedback stability

\section{Inflation as Enabling Mechanism}
Inflation rapidly expands the structure space, reducing $\eta(x)$ system-wide. This prevents global backreaction, allowing a single structure path to persist. It provides the separation necessary for directionality and structural identity.

\section{Conclusion}
The asymmetry is not accidental, but follows from a universal principle: 
\textit{Only structures that resonate with projected directionality can stabilize.}
This structural resonance bias selects matter as the stable semantic configuration of our universe.

\end{document}


\documentclass[11pt]{article}
\usepackage{amsmath, amssymb, geometry}
\geometry{margin=2.5cm}
\title{Strukturgradientenmodell zur Erklärung Dunkler Materie ohne Teilchenhypothese}
\author{}
\date{}

\begin{document}
\maketitle

\section*{Einleitung}

Dieses Dokument stellt ein alternatives Modell zur Erklärung galaktischer Rotationskurven vor, das keine Dunkle-Materie-Teilchen benötigt. Es basiert auf einer strukturellen Erweiterung des Standardmodells, die zusätzlich zur klassischen Gravitation ein semantisch interpretierbares Feld berücksichtigt.

\section*{Ausgangspunkt: Klassische Gravitationsdynamik}

Die Bahn eines Testkörpers ergibt sich im Newtonschen Rahmen aus:

\[
\frac{v^2(r)}{r} = \frac{G M(r)}{r^2}
\]

Im Außenbereich vieler Galaxien bleibt $v(r)$ jedoch nahezu konstant, obwohl $M(r)$ dort kaum noch zunimmt — ein zentrales Argument für die Existenz Dunkler Materie.

\section*{Strukturgradient als ergänzendes Feld}

Anstelle zusätzlicher Masse wird hier ein strukturwirksames Gradientenfeld $\rho(x)$ eingeführt, dessen Wirkung wie eine Kraft wirkt:

\[
\vec{F}_{\text{eff}}(r) := -\gamma \cdot \nabla \rho(r)
\]

mit:

\[
\rho(r) := \rho_0 \cdot \left[ 1 - \exp\left(-\frac{r}{r_s} \right) \right]^2
\]

und:

\[
\nabla \rho(r) = \frac{2\rho_0}{r_s} \cdot \left(1 - \exp\left(-\frac{r}{r_s} \right) \right) \cdot \exp\left(-\frac{r}{r_s} \right)
\]

\section*{Rotationsgeschwindigkeit durch strukturinduzierte Wirkung}

Diese Kraft ergibt eine effektive Bahnkurve:

\[
v(r) = \frac{\sqrt{2 \gamma \rho_0} \cdot \sqrt{r} \cdot \sqrt{1 - \exp(-\frac{r}{r_s})} \cdot \exp(-\frac{r}{2r_s})}{\sqrt{r_s m}}
\]

\section*{Beispiele realer Galaxien}

Die Funktion wurde auf reale Daten angewendet. Einige Auszüge:

\begin{itemize}
  \item \textbf{NGC 3198:} $\rho_0 = 139.87$, $r_s = 10.20\,\mathrm{kpc}$, $\gamma = 566.54$
  \item \textbf{DDO 154:} $\rho_0 = 21.71$, $r_s = 6.29\,\mathrm{kpc}$, $\gamma = 206.12$
  \item \textbf{F568-3:} $\rho_0 = 15.47$, $r_s = 7.12\,\mathrm{kpc}$, $\gamma = 441.07$
  \item \textbf{UGC 2885:} $\rho_0 = 126.88$, $r_s = 16.84\,\mathrm{kpc}$, $\gamma = 1501.73$
\end{itemize}

\section*{Erkannte Skalierungsrelationen}

\begin{itemize}
  \item $\gamma \sim \ln(1 + v_{\text{max}})$ (dynamische Kopplung)
  \item $r_s \sim \ln(1 + D)$ (skalenabhängige Reichweite)
  \item $\rho_0$ variiert mit Zentralstruktur und Masse
\end{itemize}

\section*{Ursprüngliche Hypothese: Universelle Wirkungsstruktur}


Dieses Modell wurde ursprünglich aus einer erweiterten Wirkungsfunktion entwickelt:

\[
\mathcal{S}_{\text{universum}} := \int_{\mathcal{M}} \left[
\mathcal{L}_{\text{phys}} + \alpha \|\nabla Z(x)\|^2 + \beta \|\nabla \eta(x)\|^2 + \gamma \|\nabla \rho(x)\|^2
\right] \, d^4x
\]

Dabei bezeichnen die zusätzlichen Felder:

\begin{itemize}
  \item $Z(x)$: Ein skalare Feldfunktion, die eine gerichtete semantische Zielstruktur beschreibt.
  Sie wirkt wie ein Potenzial, das Systemausrichtung und Zustandspräferenz modelliert (ähnlich einem Attraktor in dynamischen Systemen).
  
  \item $\eta(x)$: Ein Rückkopplungsmaß, das die strukturelle Selbstbeeinflussung eines Punktes beschreibt.
  Hohe Werte entsprechen instabilen oder chaotischen Regionen; formal vergleichbar mit einem nichtlinearen Verstärker oder "Reflexionspunkt".

  \item $\rho(x)$: Die Informationsstruktur bzw. semantische Dichte. Sie beschreibt die lokal wirksame strukturelle Komplexität oder Systembedeutung.
  In diesem Modell ist sie direkt für die beobachtbare Wirkung verantwortlich.
\end{itemize}

Dieses Dokument fokussiert sich ausschließlich auf die Wirkung von $\rho(x)$.  
Bei Interesse stellen wir gerne ein begleitendes Fachpapier zur Verfügung, das tiefer auf die semantische Struktur,  
die Kopplungsterme, Feldnatur, Symmetrieeigenschaften und kosmologische Perspektiven von $Z(x)$ und $\eta(x)$ eingeht.


\end{document}

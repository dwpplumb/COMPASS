
\documentclass[11pt]{article}
\usepackage{amsmath, amssymb, geometry, hyperref}
\geometry{margin=2.5cm}
\title{Structural Resolution of Singularities via the Operator $\mathcal{M}(x)$ \\[1ex]
\large A Consistent Extension beyond Classical Divergence}
\author{David Plumb\\ \href{mailto:david.plumb1980@gmail.com}{david.plumb1980@gmail.com} \\ \href{https://github.com/dwpplumb/COMPASS}{github.com/dwpplumb/COMPASS}}
\date{}

\begin{document}
\maketitle

\section*{Abstract}
We introduce the operator $\mathcal{M}(x)$ as a structural alternative to classical divergence models in General Relativity (GR). By decomposing systems into three semantically grounded axes — information density $\rho(x)$, directional gradient $\nabla Z(x)$, and feedback sensitivity $\eta(x)$ — the operator enables stable description where standard tensorial models break down. Applications to Sagittarius A*, M87*, and 3C 273 are shown with real-valued examples. Compatibility with the classical field framework is maintained while resolving singular behaviors.

\section{Motivation and Problem Scope}
Classical GR predicts singularities: infinite curvature (e.g. black holes), undefined causal flow (e.g. Cauchy horizons), and energetic instability (e.g. early universe). Traditional tensors like $g_{\mu\nu}$ and $T_{\mu\nu}$ diverge in such regimes. A new structure-preserving description is necessary.

\section{Classical Limits and Their Breakdown}
\begin{itemize}
  \item $T_{\mu\nu} \to \infty$ near singularities
  \item $g_{\mu\nu}$ loses invertibility and predictivity
  \item Time direction collapses inside ergospheres or beyond Cauchy horizons
\end{itemize}

\section{Definition of $\mathcal{M}(x)$}
We define:
\[
\mathcal{M}(x) := \left( \rho(x), \nabla Z(x), \eta(x) \right)
\]
\begin{itemize}
  \item $\rho(x)$ approximates structural information density instead of energy divergence
  \item $\nabla Z(x)$ encodes system-internal directionality as structural goal flow
  \item $\eta(x)$ quantifies feedback-induced destabilization
\end{itemize}

\section{Structural Derivations}
\begin{itemize}
  \item $\rho(x) \sim 1/|\partial_r g_{\mu\nu}|$ replaces energy density near $T_{\mu\nu}$ divergence
  \item $\nabla Z(x)$ replaces classical geodesic flow with semantic directionality
  \item $\eta(x) \sim \int |\delta \mathcal{M}(y)/\delta \mathcal{M}(x)| dy$ models reflective instability without Hawking divergence
\end{itemize}

\section{Applications and Real Data}
We apply $\mathcal{M}(x)$ to three astrophysical objects:
\begin{enumerate}
  \item \textbf{Sagittarius A*}: high $\rho(x)$, flat $\nabla Z(x)$, stable $\eta(x)$
  \item \textbf{M87*}: extremely high $\rho(x)$, directed and flat $\nabla Z(x)$, low $\eta(x)$
  \item \textbf{3C 273}: moderate $\rho(x)$, chaotic $\eta(x)$, zero gradient due to path dominance
\end{enumerate}
All values computed symbolically and shown to be finite and structurally interpretable.

\section{Compatibility with the Standard Model}
\begin{itemize}
  \item Respects underlying GR assumptions
  \item Does not break classical field logic — merely reframes divergences
  \item Potential to interface with QFT via structured state functionals
\end{itemize}

\section{Conclusion and Outlook}
The $\mathcal{M}(x)$ operator allows a reframing of classical failure zones as structurally describable fields. This opens paths to integrate meaning-bearing state spaces into physical theory, and aligns with ongoing efforts to resolve quantum-gravitational tensions.

\bigskip
\noindent
Contact: \textbf{David Plumb} — \texttt{david.plumb1980@gmail.com} \\
GitHub Project: \url{https://github.com/dwpplumb/COMPASS}

\end{document}
